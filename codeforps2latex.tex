<DOCTYPE>
	<style>
	
.btn-group button {
  background-color: white;
  border: 3px solid black; 
  color: black; 
  padding: 10px 24px; /
  cursor: pointer; 
  float: left; 
}

.btn-group button:not(:last-child) {
  border-right: none;
}


.btn-group:after {
  content: "";
  clear: both;
  display: table;
}


.btn-group button:hover {
  background-color: lightgrey;
}

.container { 
  height: 200px;
  position: relative;
  width: 100%
  
}

.center {
  position: absolute;
  top: 50%;
  left: 50%;
  -ms-transform: translate(-50%, -50%);
  transform: translate(-50%, -50%);
}
.header{
	
padding: 30px;
text-align: center;
background: black;
color: white;
font-size: 30px;
font-family = "Courrier New", monospace;
margin-bottom: 100px;

	
}


</style>
<title> Team 3 Sensors</title>
<meta name ="viewpoint" content="width=device-width, initial-scale=1"/>

	
	
	</style>
	<div class = "header">

	<h2 style ="font-family: Arial"> Sensor Dashboard</h2>
	<p style ="font-family: Arial"> William Trolinger, Dalton Oxford, Jake Parnell</p>
	
	
	</div>
	

<body>
		
		
	<table style =  "text-align: center"id = "sensorTable" width = 100%>
		
		<tr>
			
			<th style ="font-family: Arial " width ="100px" id = "row_1col1">Tempurature</td>
			<th  style ="font-family: Arial" width ="100px" height = "50px" id = "row_1col2">Object Detected</td>
			
			
		</tr>
		<tr>
			<td  style ="font-family: Arial text-align: center font-size: 30px" width ="100px" height = "50px" id = "row_2col1"> </td>
			<td  style ="font-family: Arial text-align: center" width ="100px" height = "50px" id = "row_2col2"> </td>
		
		</tr>
	
	</table>
	
	
</body>


	
	
<body>  
	
	<div class= "btn-group" width = 50% >
		
		<div class="container">
			<div class="center" width = 20%>
		
  
  <button style ="font-family: Arial" width = "40px" id = "buzButton" onclick = "buzzButton()" type = "button">Buzzer</button>
	<button style ="font-family: Arial"id = "ledButton" onclick = "actionLedLight()" type = "button">   Light</button>
  
  </div>
</div>
	
	
	
	</div>
	
	
</body>

<script>


	// SENSOR scdedule remain a key piece of the
	
	window.setInterval(
		function(){
			console.log("Reading on second intervale");
			try { 
				
				
				pullTempature();
			}catch(e) {
				
			}
			
			actionDetechObject();
			
		}
		
	,1000);
	
	
	function setupMain() { 
		
		console.log("Main Set up Running");
		
		
		// setting up pull request on intervale in order to make sure that getting the right information from the 
		// from the program and make sure everthing is good in the program
		//setTimeout(pullTepmature, 3000);
	}
	
	
	
	function pullDetechObject()  { 
		
			jsonObject = fetch("detectObject").then(function (resp){return resp.json();}).then(function (data){
				
					
				
					stringBack = data.ObjectDetech; 
					return stringBack;
				
					
				}
			
			
			
			
			);
			
			
	
		
			
		
	}
	
	
	function setObjectDetech(detectation){ 
		
		
		document.getElementById("row_2col2").innerHTML = detectation;
		
	}
	
	
	
	function actionLedLight() { 
			
			
			fetch("LedLight").then(function (resp) {return resp.json()}).then(function (data){
				
				
					
				
			});
		
		
	}
	
	
	
	function actionDetechObject() { 
		
		
		/*
		function pullTempature() { 
		
		
			fetch("getTempature").then(function (resp){ return resp.json()}).then(function(data) { 
				
				
				tempature = data.Tempature; 
				console.log(tempature); 
				setTextTempature(tempature);
				
				
			});
		
	}
		
		*/
		fetch("detectObject").then(function (resp) {return resp.json()}).then(function (data) {
			
			console.log("Here in the code");
			returnBackReading = data.ObjectDetech; 
			
			console.log(returnBackReading);
			document.getElementById("row_2col2").innerHTML = returnBackReading;
			
		} 
		
		
		
		
		);
		
		
	}
	
	function pullTempature() { 
		
		
			fetch("getTempature").then(function (resp){ return resp.json()}).then(function(data) { 
				
				
				tempature = data.Tempature; 
				console.log(tempature); 
				setTextTempature(tempature);
				
				
			});
		
	}
	
	// tempature is going to be string
	function setTextTempature(tempature){ 
		
		//document.getElementById("tempature").innerHTML = "Tempature: " +  tempature + ", c";
		document.getElementById("row_2col1").innerHTML = tempature;
		
	}
	
	
	

	
	
	
	
	function buzzButton(){ 
		
		try  { 
			
			
			fetch("buzzBuzz").then(resp => function (resp){return resp.json()});
			console.log("Buzz buz");
		}catch(e) { 
			
				console.log("no buzz buz sad times!!");
		}
	}
	
	
	function pullObject() { 
		
			try {
				console.log("Object detech");
				
				
			}catch(e) { 
				
				
				
			}
		
	}
	
	
	/* 
		Tempature function and code is given below for the project so goin g
	
	*/
	
	
	function buttonPress() { 
		
			console.log("Button was press running test");
			document.getElementById("theButton").innerHTML = "Bet";
			getInformationFromServer();
			
	}
	
	function getInformationFromServer() { 
		
		fetch("getBack").then(function (resp){return resp.json()}).then(function(data)
		{
			
			console.log(data)
			try { 
				
				
				
				
				JSON.parse(data);
			
			
			}catch(e) { 
				
					console.log("Json parsing dont work at all so no good");
				
			}
			
			name = data.Name
			console.log(name);
			
			
			updateText(name);
			
			
			
			
		});
		
		
			
		
		
	}
	
	function updateText(tempNum){ 
		
		document.getElementById("tempature").innerHTML = "Temperature: " + tempNum + " c";
		document.getElementById("sensorTable_point3").innerHTML =tempNum;
		
		
	}

	
</script>
<style>
	#sensorTable, td,tr{ 
		
		border: 1px solid black;
		
	}

</style>

########################################################################################################################


import PCF8591 as ADC
import RPi.GPIO as GPIO
import time
import math
import os

#cherry-py
import cherrypy



index = 0;
colors = [0xFF00, 0x00FF, 0x0FF0, 0xF00F]
pins = (13, 22)

#web server and main functioning code
#######################################################

def main():
	
	
	
	print("Here in the code");
	print("Current pages directory is going to be ",getPagesDirectory());
	print("\n\nSt arting up web server...");
	
	# simple start for the web-server
	cherrypy.quickstart(webFrame() );
	

	return;

class webFrame(object): 

	

	@cherrypy.expose
	def index(self):
		return loadPage(simplePageGrab("Page1.html"));
	
	
	
	
	# NOT NEEDED was just testing from the past
	@cherrypy.expose
	@cherrypy.tools.json_out()
	def getBack(self):
		
		#
		print("request");
		
		
		# Getting back a string from server whcih send back with in order to make sure everthing is 
		# everthing is working the right
		tempReadValue = tempRead();
		print("\n\nTemp Value: " , tempReadValue , "\n\n");
		return {"Name": tempReadValue};
	
	
	@cherrypy.expose
	@cherrypy.tools.json_in()
	def buzzBuzz(self):
		
		print("buzz buzz");
		
		buzzerOn();
		
		return;
	
	########################	
	#@cherrypy.expose
	#@cherrypy.tools.json_in()
	#def lightsOn(self):
		
	#	print("let there be light");
		
	#	lightOn   ();
		
	#	return;
		
	# Get Back the Tempature for the program
	@cherrypy.expose
	@cherrypy.tools.json_out()
	def getTempature(self): 
		
		tempReadValue = tempRead(); 
		
		
		return {"Tempature" : tempReadValue};
	
	
	
	@cherrypy.expose
	@cherrypy.tools.json_out()
	def detectObject(self):
		
		print("detecting object in the");
		return{"ObjectDetech": str(checkDetech() ) };


	@cherrypy.expose
	@cherrypy.tools.json_out()
	def LedLight(self):
		
		global index;
		print("running");
		setColor(colors[index]);
		index = index + 1;
		index = index % 4;
		print("led index:", index );
		
		return {"Key":"thing"}
		
	
# 
#######################################################

#FILE HANDLING
#######################################################

# get the current directory of the program you are running in order to make sure 




def loadPage(pageFile): 
	
	pageFileStream = open(pageFile, "r"); 
	returnPage = pageFileStream.read(); 
	pageFileStream.close(); 
	
	
	
	
	return returnPage;



# simples add the Pages to the before of the pageName you are passing so you dont have to write all that stuff 
# in order to find the html file you are looking for
def simplePageGrab(pageName):
	
	return os.path.join(getPagesDirectory() , pageName);
	
def getCurrentDirectory():
	return os.getcwd();
	
	
def getFiles(directoryFile):
	
	fileList = os.listdir();
	
	
	index = 0;
	tempPath = directoryFile;
	for element in fileList:
		tempPath = os.path.join(directoryFile, element); 
		fileList[index] = tempPath;
		index = index +1;
	return fileList;
	
	


def getPagesDirectory(): 
	
	currentWorkingDirectory = getCurrentDirectory(); 
	fileList = getFiles(currentWorkingDirectory);


	lookingFor = os.path.join(currentWorkingDirectory , "Pages");
	
	
	for element in fileList:
		if (element == lookingFor ):
			return element;
		
	
	
	return;
		
	

########################################################
# SENSOR CODE
########################################################
DO = 29
buzzerPin = 12; # buzzer pin out
objectDetectPin = 11; # object detect pinout
buzzer_object = None;


p_R = None;
p_G = None;
#LEDG = 13
#LEDR = 22

#buzzerPin = 12
#GPIO.setmode(GPIO.BCD)

####################################
# BUZZER SET-UP AND PINOUTS
#BUZZER = 12;
#GPIO.setup(buzzerPin, GPIO.OUT);
#buzzer = GPIO.PWM(buzzerPin, 1000);



def setup():
	
	global colors, buzzer_object, p_R, p_G;
	
	GPIO.setmode(GPIO.BOARD); 
	ADC.setup(0x48)
	GPIO.setup(DO, GPIO.IN);	  
	GPIO.setup(buzzerPin, GPIO.OUT);
	
	GPIO.setup(objectDetectPin, GPIO.IN);
	
	buzzer_object = GPIO.PWM(buzzerPin, 1000);
	
	
	
	# LED light
	GPIO.setup(pins, GPIO.OUT) # Set pins' mode is output
	GPIO.output(pins, GPIO.LOW) # Set pins to LOW(0V) to off led

	p_R = GPIO.PWM(pins[0], 2000) # set Frequece to 2KHz
	p_G = GPIO.PWM(pins[1], 2000)
	
	p_R.start(0); 
	p_G.start(0);
	setColor(colors[1]);
	#GPIO.setup(LEDG, GPIO.OUT)
	#GPIO.setup(LEDR, GPIO.OUT)
	#GPIO.output(LEDR, 1)
	#GPIO.setup(buzzPin, GPIO.OUT)
	#buzzer = GPIO.PWM(buzzPin, 1000)



def map(x, in_min, in_max, out_min, out_max):
	return (x - in_min) * (out_max - out_min) / (in_max - in_min) + out_min;
	


def setColor(col): # For example : col = 0x1122
	
	R_val = col >> 8
	G_val = col & 0x00FF


	R_val = map(R_val, 0, 255, 0, 100)
	G_val = map(G_val, 0, 255, 0, 100)

	p_R.ChangeDutyCycle(R_val) # Change duty cycle
	p_G.ChangeDutyCycle(G_val)


	

def Print(x):
	if x == 1:
		print ('')
		print ('***********')
		print ('* Better~ *')
		print ('***********')
		print ('')
	if x == 0:
		print ('')
		print ('***********')
		print ('* HOT~    *')
		print ('***********')
		print ('')





# Reading from the program are just wrong and this could can be because of coding problems or other problems with the program but make a difference how
# 1. if it is the sensor we are going to just going to have to remove the sensor in order to make sure everthing is working the right this is going
# 2. coding problem if so wrong input and function or just wrong math is being done in the program need to be check 
def tempRead(): 
	
		# pre-set before the for loop in order to make sure everthin is good with the program remain a key piece of the 
		
		status = 1;
		tmp = 1;
		
		
		analogVal = ADC.read(1)
		Vr = 5 * float(analogVal) / 255
		Rt = 10000 * Vr / (5 - Vr)
		print("Value of Rt ---->",Rt);
		temp = math.log(Rt/10000) / 3950;
		temp = temp + (1 /(273.15 + 15));
		temp = 1 / temp
		#temp = 1 / ((math.log(Rt /10000)) / 3950) /1  + (1 /(273.15 + 15))
		#temp = 1/(((math.log(Rt / 10000)) / 3950) + (1 / (273.15+25)))
		temp = temp - 273.15
		print ('temperature = ', temp, 'C')

		# For a threshold, uncomment one of the code for
		# which module you use. DONOT UNCOMMENT BOTH!
		#################################################
		# 1. For Analog Temperature module(with DO)
		tmp = GPIO.input(DO);
		# 
		# 2. For Thermister module(with sig pin)
		if temp > 33:
			tmp = 0;
		elif temp < 31:
			tmp = 1;
		
		
		
		#calling right here
		#################################################
		#if temp > 50:
		#	lightOn()
		#	buzzerOn()
		#else:
		#	lightOff()
		
		##################################################
		
		
		
		
		
		
			
		
		if tmp != status: 
			print(tmp)
			status = tmp
			
		#time.sleep(0.2);
		return str(temp);
		#if tmp != status:
		#	print(tmp)
		#	status = tmp

		#time.sleep(0.2)

	

def checkDetech():
	
	
	index = 0;
	while index < 10000: 
		if GPIO.input(objectDetectPin): 
			
			return False; 
			
		else: 
			return True;
	
	
	
	

def loop():
	status = 1
	tmp = 1
	while True:
		analogVal = ADC.read(1)
		Vr = 5 * float(analogVal) / 255
		Rt = 10000 * Vr / (5 - Vr)
		print("Value of Rt ---->",Rt);
		temp = math.log(Rt/10000) / 3950
		temp = temp + (1 /(273.15 + 15));
		temp = 1 / temp
		#temp = 1 / ((math.log(Rt /10000)) / 3950) /1  + (1 /(273.15 + 15))
		#temp = 1/(((math.log(Rt / 10000)) / 3950) + (1 / (273.15+25)))
		temp = temp - 273.15
		print ('temperature = ', temp, 'C')

		# For a threshold, uncomment one of the code for
		# which module you use. DONOT UNCOMMENT BOTH!
		#################################################
		# 1. For Analog Temperature module(with DO)
		tmp = GPIO.input(DO);
		# 
		# 2. For Thermister module(with sig pin)
		if temp > 33:
			tmp = 0;
		elif temp < 31:
			tmp = 1;
		#################################################

		if tmp != status:
			Print(tmp)
			status = tmp

		time.sleep(0.2)
		
def lightOn():
	GPIO.output(LEDR, 0)
	GPIO.output(LEDG, 1)
	
def lightOff():
	GPIO.output(LEDR, 1)
	GPIO.output(LEDG, 0)
	
def buzzerOn():
	
	global buzzer_object;
		
	buzzer_object.start(10)
	buzzer_object.ChangeFrequency(440)
	time.sleep(1)
	buzzer_object.stop()
	


if __name__ == '__main__':
	try:
		setup()
		#loop()
		main();
		
		
	except KeyboardInterrupt: 
		# This was the origal point in the code and that how I had it before in the 
		# program
		#pass m
		GPIO.cleanup()
		print("something here in the code");    
		



